\section{Discussion}
\label{ch:concl}
\noindent	

Throughout both presented projects, a lot of linear algebra was necessary in order to finalize them. As such, it is safe to say that goal 2 is more than well achieved. Likewise, during the creation of the raytracer some understanding of physical concepts was also necessary to implement the different material properties. This satisfies the third goal. Lastly, the third course block presented some knowledge on the topic of OpenGL, which has briefly been summarized in the theory chapter, in turn completing the first goal.

\iffalse
The conclusion/discussion (choose a heading) is a separate chapter in which the results are analysed and critically assessed. At this point your own conclusions, your subjective view, and explanations of the results are presented.

If this chapter is extensive it can be divided up into more chapters or sub-chapters i.e. one analysis or discussion chapter with explanations of and critical assessment of the results, a concluding chapter where the most important results and well supported conclusions are discussed and to sum it up a chapter with suggestions for further research in the same area. In this chapter it is of vital importance that a connection back to the aim of the survey is made and thus the purpose is pointed out in a summary and analysis of the results. 

In this chapter you should also include answers to the following questions: What is the project's news value and its most vital contribution to the research or technology development? Have the project's goals been achieved? Has the task been accomplished? What is the answer to the opening problem formula? Was the result as expected? Are the conclusions general, or do they only apply during certain conditions? Discuss the importance of the choice of method and model for the results. Have new questions arisen due to the result?

The last question invites the possibility to offer proposals to others relevant research, i.e. proposal points for measures and recommendations, points for continued research or development for those wishing to build upon your work. In technical reports on behalf of companies, the recommended solution to a problem is presented at this stage and it is possible to offer a consequence analysis of the solution from both a technical and layman perspective, for example regarding environment, economy and changed work procedures. The chapter then contains recommended measures and proposals for further development or research, and thus to function as a basis for decision-making for the employer or client.
\fi

\subsection{Course discussion}

The course was overall quite good, if albeit a little bit concise. Certain fields were not explained in incredible detail, but instead left for the student to figure out. This could very well have been an intentional decision from the authors of the course in order to ''force'' the student to grasp certain concepts. The foundation laid out during the course regarding certain math or graphics-related concept was rock-solid, however, making the course very suitable as an introduction to advanced computer graphics.

\subsection{Future Work}
\label{ch:concl:future-work}

A lot of different improvements can be done in the two projects, particularly the raytracer. Examples for improvements regarding the raytracer include, but are not limited to:

\begin{itemize}
	\item Adding support for loading/saving scenes
	\item Adding support for other shapes than boxes and spheres.
	\item Adding support for different materials.
	\item Adding support for light sources within the scene.
	\item As well as further optimizing the general raytracing algorithm, as the current version of the raytracer takes several seconds to render a scene even with multithreading active.
\end{itemize}
