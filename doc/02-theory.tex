\section{Theory}
\label{ch:theory}
\noindent
This chapter will present theory related to the project.

\subsection{Graphics Processing Unit (GPU)}

The GPU works as a complementary workhorse to the CPU. Whereas the CPU remains flexible in terms of what it is able to compute, the GPU is straight up a stronger version of a CPU when regarding raw computing power. The majority of GPUs enjoy working with triangles, partially due to it being a mathematically pleasant shape to work with, as well as it being a very flexible shape geometrically - most other shapes can be approximated by using the correct amount of triangles.\cite{gpu}

\subsection{OpenGL}

The Open Graphics Library has been the most widely adopted graphics API within the industry for quite some time now, regardless of given the context of a 3D or 2D environment. The first version was released in 1992, with the latest version being released in 2017.\cite{opengl}

\subsubsection{OpenGL Shading Language (GLSL)}

The GLSL programming language is the language used to interact with and manipulate the graphics pipeline. The main benefit gained from using a specific language for this task is that developers no longer need to write hardware-specific or OS-specific code in order to program the GPU.\cite{glsl}

\subsection{Linear Algebra}

Linear Algebra is the subset of mathematics which, amongst other things, discusses the presentation of different kinds of geometry and its transformations, including how geometry can be mapped from one location to another. Two central concepts to the field of linear algebra include the vector: an $N$ -dimensional scalar, as well as the matrix: an $N\cdot M$ -dimensional scalar.

\subsubsection{Dot product}

The dot product is an operation between two vectors ($a$ and $b$) of equal dimension which produces a scalar. A more formal definition would look something like:

\begin{figure}[H]
	\begin{center}
		$a \cdot b = \sum_{i=1}^{n}{a_ib_i = a_1b_1 + a_2b_2 + ... a_nb_n}$
		\caption{The definition of the dot product.}
	\end{center}
\end{figure}

\subsubsection{Cross product}

The cross product is an operation between two vectors ($a$ and $b$) of equal dimension which produces a third vector which is orthogonal to both prior vectors. A more formal definition would look something like:

\begin{figure}[H]
	\begin{center}
		$a \times b = \left\vert\left\vert a \right\vert\right\vert \left\vert\left\vert b \right\vert\right\vert sin(\theta) n$
		\caption{The definition of the cross product.}
		\label{cross-formula}
	\end{center}
\end{figure}

In \textit{figure~\ref{cross-formula}} $\theta$ represents the angle between $a$ and $b$ and $n$ represents the unit vector of the plane which contains $a$ and $b$ in the direction which is given by the right hand rule. A more practical definition can be given if the dimension of the vectors is given, such as:\cite{cross-howto}

\begin{figure}[H]
	\begin{center}
		$ u = (u_x, u_y, u_z), v = (v_x, v_y, v_z)$
		$ u \times v = ((u_y v_z - u_z v_y), (u_x v_z - u_z v_x), (u_x v_y - u_y v_x))$
		\caption{The second, sometimes more practical definition of the cross product.}
		\label{cross-formula-2}
	\end{center}
\end{figure}

\subsubsection{Matrix Multiplication}

\begin{figure}[H]
	\begin{center}
$A_{m,n} = 
\begin{pmatrix}
a_{1,1} & a_{1,2} & \cdots & a_{1,n} \\
a_{2,1} & a_{2,2} & \cdots & a_{2,n} \\
\vdots  & \vdots  & \ddots & \vdots  \\
a_{m,1} & a_{m,2} & \cdots & a_{m,n} 
\end{pmatrix}$
	\end{center}
\end{figure}

\iffalse
In the report's theory study, sometimes called Related work, there may be additional facts required for the reader's understanding of the report. At this point a summary of background material in the area should be provided, i.e. standards, scientific articles, books, magazines, documents on the web, technical reports and user manuals. Explain pedagogically with clear examples and many illustrations.

It should be demonstrated that you have an awareness of the context and the background of your work in addition to that carried out by you within the project. Explain the aim of the technology that you describe, and not only how the technology works. For D-level you should display an awareness of the key research within the area, in order to ensure that your work has a certain news value. However it is vital that you do not deviate too much from your research problem.

Your assignment is not to write a textbook. It is important to find an appropriate balance between related work and your own results. The theory study should only constitute a minor portion of a thesis.

Instead of “Theory” or “Related work”, the heading may very well be a specific topic, for example “The GSM standard” or ”A survey on the research field of X".

If the theoretical study section is rather brief then it is possible to include it within the Introduction chapter.

If your method is to undertake a critical literature study you normally do not have to have a separate chapter with background material because all sources you refer to are summarized in the results chapter. Your criticism of the sources and the arguments for your personal opinions are thus placed in the concluding chapter.

\subsection{Definition of terms and abbreviations}
\label{ch:theory:definitions}
Terms and abbreviations that are important for the reader's continued understanding are explained in this chapter. The first time you insert text that uses a concept or an abbreviation you should also explain it, even if it is already defined in the terminology section. The concept is typed using the italic style.

The first time an abbreviation (abbr.) is used it is typed within the parenthesis after its explanation, as illustrated in this sentence.

\subsubsection{Example of level 3 heading}
\label{ch:theory:level3-heading}
Avoid too many heading levels.

\subsection{To review or quote}
\label{ch:theory:review:quote}
You review when you reproduce content using your own words.

Example: Forslund [4] recommends more informative headings be used in technical reports and that one should, in particular, provide important information in the sub headings.

You quote when you literally reproduce a phrase, a sentence or paragraph. Quotations under 50 words are to be placed within quotation marks. To quote Strömqvist could be a suitable illustration in this context: “It may be difficult to write, but it is also fun” \cite{stomquist}.

Quotations over 50 words should be reproduced in the form of block quotations. The text block is centered on the page without quotes and in small caps. The source is stated in direct connection to the block quotation.

Normally you review instead of using quotes. You can use direct quotations if you wish to reproduce established definitions of concepts, which you believe an author has formulated himself in a particularly suitable manner, when you require aid of an authority, or when you wish to demonstrate that an author is wrong.

\subsection{References and source references}
Kindly observe! To reproduce a text without stating its source is to be considered as plagiarism and is thus defined as serious cheating.

A list of references is placed at the end of the report in order to give the reader overall information regarding all reviewed sources, quotes or for any other reasons that you need to refer to in the text. The sources should be carefully stated so that the reader can check if it is available in libraries or on the internet. Sometimes it might be that verbal sources and other correspondence are included in the source list, but this is unusual in technical reports.

Refrain from using less trustworthy sources, instead stick to using material written by authorities in the subject matter. Private sites and exam papers are seen as having a low reliability as sources. This is especially true if the exam paper is of a lower level then your own paper.

Use only sources in the list that you refer to or quote in the continuous text. All sources that are used in the source list should be linked to the report through reference in the continuous text, according to the Vancouver-system, which commonly occurs in reports regarding technical matters.

According to the Vancouver-system the source list is arranged in the same order as the sources appear in the continuous text, the source reference is to be stated in the text with a figure within square brackets, i.e. \cite{dataterm-kth} or \cite{eriksson-2001}, \cite{stomquist}. They should also be stated in this order in the source list. Examples of source reference: According to Eriksson \cite{eriksson-2001}  dynamic SFNs can provide significant performance improvements.
\fi
